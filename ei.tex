\documentclass[a4paper]{article}

\usepackage{tocloft}
\usepackage{amsmath}
\usepackage{graphicx}
\usepackage{etoolbox}
\usepackage{hyperref}
\usepackage[utf8]{inputenc}
\usepackage[brazil]{babel}
\usepackage[colorinlistoftodos]{todonotes}

\addtocontents{toc}{\cftpagenumbersoff{section}}

\title{Espiritualidade Instrumental}

\author{Jesus de Nazaré e Paulo de Tarso}

\date{\today, versão 0.17- alfa}

\usepackage{parskip}
\setlength{\parindent}{20pt}

\makeatletter
  \pretocmd{\chapter}{\addtocontents{toc}{\protect\addvspace{15\p@}}}{}{}
  \pretocmd{\section}{\addtocontents{toc}{\protect\vspace{-3mm}}}{}{}
\makeatother

\begin{document}

\maketitle

\begin{abstract}
Argumentações e apontamentos de aspectos instrumentais da espiritualidade para
aproveitamento de todos os indivíduos.
\end{abstract}

\tableofcontents

\section{Ponto de vista e propósito deste texto}

\qquad Este texto parte tanto da premissa de incompletude da representação de
mundo de qualquer afirmação, quanto do princípio de mundo aberto, segundo o qual
um fato pode ser verdade independente do conhecimento ou não de alguma fonte.

O propósito é difundir o aspecto instrumental e útil da espiritualidade para os
indivíduos praticantes. A motivação é desmistificar o assunto e colocar o
amadurecimento no campo que merece: de aproveitamento. Independente da crença do
beneficiário e mesmo que em nada creia.

\begin{quotation}
  Portanto ide, fazei discípulos de todas as nações, batizando-os em nome do
  Pai, e do Filho, e do Espírito Santo; \textit{Mateus 28:19}
\end{quotation}

Ressonâncias corruptas, como preconceituosas ou corporativistas, são amplamente
observadas na esfera religiosa. É assumido aqui que tal problema é fruto de
dificuldades mais gerais. Uma boa comparação é a corrupção da indústria de
alimentos, ao passo que alimentar-se não é a fonte desta mesma dificuldade, mas
sim características humanas mais gerais.

\section{A espiritualidade como vantagem evolutiva}

\qquad A atividade espiritual é observada com facilidade em quase todas as
culturas, portanto é traço preservado pela evolução. Há vantagens para o
indivíduo e/ou alguma escala da população. Dentre as causas desta vantagem
evolutiva estão:

\begin{itemize}
  \item treinamento neural e outras elaborações mentais, como descrito na
  parte~\ref{met}.

  \item Estabelecimento de convenções (repertórios comuns de informação).

  \item Preservação e propagação de comportamentos vantajosos evolutivamente.
  Por exemplo: se alguém não matar ou roubar (seja por princípios religiosos ou
  não), terá menos chance de ser assassinado ou prejudicado e mais chance de ser
  ajudado e até defendido em algum confronto.
\end{itemize}

\qquad Os sapiens saíram do leste da África pela segunda vez com túmulos,
pintando, com capacidades sociais e mentais mais avançadas, porque tinham
descoberto a tecnologia espiritual.

\section{A espiritualidade como método de conquista da realidade}\label{met}

\qquad É abstração para treinamento neural. É repositório ancestral de meios
para ampliar a percepção e consciência dos atos e pensamentos, ou seja, para
trazer à consciência aspectos de antemão apenas acessados pelo inconsciente.

A natureza fragmentada do pensamento humano coloca a concepção da unidade como
algo transcendental, ligada às práticas meditativas e assumindo plena
visibilidade na figura do Deus monoteísta. Neste sentido, a concepção da figura
onipotente (onipresente e onisciente) é um curto-circuito de libido, permitindo
acessos profundos e eficientes ao inconsciente, muitas vezes com prazeres
intensos e transes.

\begin{quotation}
  Porque a loucura de Deus é mais sábia do que os homens; e a fraqueza de Deus é
  mais forte do que os homens. \textit{1 Coríntios 1:25}
\end{quotation}

\section{A espiritualidade como tecnologia ingênua}

\qquad Sistemas formais sem uma definição axiomática única são por vezes
chamados ``ingênuos''. Neste mesmo sentido, os símbolos, rituais e
sistematizações de esferas espirituais caminham na direção de formalização e
estabilização de ideias sem que haja uma definição axiomática única ou mesmo
nítida do sistema considerado.

Essa formalização remete ao fundamento, daquele que não se vangloria de sí
próprio mas obedece à uma entidade única e singular.

O desenvolvimento da espiritualidade como tecnologia ingênua é intenso desde sua
gênese com os sapiens. Suas bases empíricas e de transmissão para descendentes
são perfeitamente racionalizáveis e passíveis de aproveitamento consciente.

\section{Aproveitamento eficiente do arcabouço}

\qquad As próprias tradições possuem instruções para melhor aproveitamento do
legado espiritual. Dentre estas, constam:

\begin{itemize}
  \item Prática: a alocação de tempo para atividades espirituais desenvolve
  capacidades fluentes para seu aproveitamento.

  \item Aculturação: o conhecimento vinculado às tradições espirituais foi
  decantado por toda ancestralidade. É vantajoso o aprendizado constante deste
  legado.

  \item Conhecimento de diferentes tradições: facilita reconhecer instrumentais,
  pois existem constâncias. Ao mesmo tempo, evitam ressonâncias corporativistas
  e preconceituosas.
\end{itemize}

\end{document}
