\documentclass[a4paper]{article}

\usepackage{hyperref}
\usepackage[portuguese]{babel}
\usepackage[utf8]{inputenc}
\usepackage{amsmath}
\usepackage{graphicx}
\usepackage[colorinlistoftodos]{todonotes}

\usepackage{tocloft}
\addtocontents{toc}{\cftpagenumbersoff{section}}


\title{Espiritualidade Instrumental}
\author{Jesus de Nazaré e Paulo de Tarso}

\date{\today, versão 0.14 - beta}

\usepackage{etoolbox}

\makeatletter
\pretocmd{\chapter}{\addtocontents{toc}{\protect\addvspace{15\p@}}}{}{}
\pretocmd{\section}{\addtocontents{toc}{\protect\vspace{-3mm}}}{}{}
\makeatother


\begin{document}
\maketitle



\begin{abstract}
Apontamentos instrumentais da espiritualidade para aproveitamento de todos os indivíduos.
\end{abstract}

\tableofcontents

\section{Ponto de vista e propósito deste texto}
Este texto subentende a incompletude da representação de mundo de qualquer afirmação e também o princípio de mundo aberto, segundo o qual um fato pode ser verdade independentemente do (re)conhecimento de alguma fonte.

O propósito é difundir o aspecto instrumental e útil da espiritualidade para os indivíduos praticantes. A motivação é desmistificar o assunto e colocar o amadurecimento no campo que merece: de aproveitamento, de usufruto, independente da crença do beneficiário e mesmo que em nada creia.

Ressonâncias corruptas, como preconceituosas ou corporativistas, são amplamente observadas na esfera religiosa. Tal problema é fruto de
dificuldades mais gerais.
Considere a da indústria de alimentos: há pessoas sem alimentação e outras mau alimentadas.
Alimentar-se não é a fonte destes problemas,
mas sim características humanas mais gerais,
tais como a ganância e a gula.

\section{A espiritualidade como vantagem darwinista}
A atividade espiritual é observada em todas as culturas, portanto é traço preservado pela evolução biológica.
Há vantagens para o indivíduo e/ou alguma escala da população. Dentre as causas desta vantagem evolutiva estão:
\begin{itemize}
  \item treinamento neural e outras elaborações mentais: conceitos abstratos, ritos, músicas, como descrito na parte~\ref{met}. A meditação, por exemplo, aumenta o hipocampo, módulo cerebral central para a memória e diversas outras funções-chave.
    \item Estabelecimento de convenções (repertórios comuns de informação), assim como a hospitalidade com estrangeiros e propensão ao perdão.
    \item Preservação e propagação de comportamentos vantajosos. Por exemplo: se alguém não matar, roubar ou desejar o cônjuge alheio (seja por princípios religiosos ou não), terá menos chances de ser assassinado ou prejudicado e mais chances de ser ajudado e até defendido em algum confronto.
\end{itemize}

Ao descobrirem a tecnologia espiritual, os sapiens
saíram do leste da África pintando,
com ritos funerários elaborados,
capacidades sociais e mentais mais avançadas,
e aculturaram os 5 continentes.

\section{A espiritualidade como método de conquista da realidade}\label{met}
É abstração para treinamento neural. É repositório ancestral de meios para ampliar a percepção dos atos e pensamentos, ou seja, para trazer à consciência aspectos de antemão acessados apenas pelo insconsciente.

A natureza fragmentada do pensamento humano coloca a concepção da unidade como algo transcendental, ligada às práticas meditativas e assumindo plena visibilidade na figura do Deus monoteísta. Neste sentido, a concepção da figura onipotente (onipresente e onisciente) é um curto-circuito de ideações, de libido, permitindo acessos profundos e eficientes ao inconsciente, muitas vezes com prazeres intensos e transes.

\section{A espiritualidade como tecnologia ingênua}
Sistemas formais sem uma definição axiomática única são por vezes chamados ``ingênuos''. Neste mesmo sentido, os símbolos, rituais e sistematizações de esferas espirituais caminham na direção de formalização e estabilização de ideias sem que haja uma definição axiomática única ou mesmo nítida do sistema considerado.

O desenvolvimento da espiritualidade como tecnologia ingênua é intenso desde sua gênese com os sapiens e demais hominídeos. Suas bases empíricas e de transmissão para descendentes são perfeitamente racionalizáveis e passíveis de aproveitamento consciente.

\section{Aproveitamento eficiente da espiritualidade}
As próprias tradições possuem instruções para melhor aproveitamento do legado espiritual. Dentre estas, constam:
\begin{itemize}
    \item Prática: a alocação de tempo para atividades espirituais desenvolve capacidades fluentes para seu aproveitamento.
    \item Aculturação: o conhecimento vinculado às tradições espirituais foi decantado por toda ancestralidade. É vantajoso o aprendizado constante deste legado.
    \item Conhecimento de diferentes tradições: facilita reconhecer instrumentais, pois existem constâncias. Ao mesmo tempo, evitam ressonâncias corporativistas e preconceituosas.
\end{itemize}

\end{document}

