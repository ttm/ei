\documentclass[a4paper]{article}

\usepackage{tocloft}
\usepackage{amsmath}
\usepackage{graphicx}
\usepackage{etoolbox}
\usepackage{hyperref}
\usepackage[utf8]{inputenc}
\usepackage[brazil]{babel}
\usepackage[colorinlistoftodos]{todonotes}

\addtocontents{toc}{\cftpagenumbersoff{section}}

\title{Espiritualidade Instrumental}

\author{Jesus de Nazaré e Paulo de Tarso}

\date{\today, versão 0.14- alfa}

\makeatletter
  \pretocmd{\chapter}{\addtocontents{toc}{\protect\addvspace{15\p@}}}{}{}
  \pretocmd{\section}{\addtocontents{toc}{\protect\vspace{-3mm}}}{}{}
\makeatother

\begin{document}

\maketitle

\begin{abstract}
Argumentações e apontamentos de aspectos instrumentais da espiritualidade para
aproveitamento de todos os indivíduos.
\end{abstract}

\tableofcontents

\section{Ponto de vista e propósito deste texto}

Este texto parte tanto da premissa de incompletude da representação de mundo de
qualquer afirmação, quanto do princípio de mundo aberto, segundo o qual um fato
pode ser verdade independente do conhecimento ou não de alguma fonte.

O propósito é difundir o aspecto instrumental e útil da espiritualidade para os
indivíduos praticantes. A motivação é desmistificar o assunto e colocar o
amadurecimento no campo que merece: de aproveitamento. Independente da crença do
beneficiário e mesmo que em nada creia.

Ressonâncias corruptas, como preconceituosas ou corporativistas, são amplamente
observadas na esfera religiosa. É assumido aqui que tal problema é fruto de
dificuldades mais gerais. Uma boa comparação é a corrupção da indústria de
alimentos, ao passo que alimentar-se não é a fonte desta mesma dificuldade, mas
sim características humanas mais gerais.

\section{A espiritualidade como vantagem evolutiva}

A atividade espiritual é observada com facilidade em quase todas as culturas,
portanto é traço preservado pela evolução. Há vantagens para o indivíduo e/ou
alguma escala da população. Dentre as causas desta vantagem evolutiva estão:

\begin{itemize}
  \item treinamento neural e outras elaborações mentais, como descrito na
  parte~\ref{met}.

  \item Estabelecimento de convenções (repertórios comuns de informação).

  \item Preservação e propagação de comportamentos vantajosos evolutivamente.
  Por exemplo: se alguém não matar ou roubar (seja por princípios religiosos ou
  não), terá menos chance de ser assassinado ou prejudicado e mais chance de ser
  ajudado e até defendido em algum confronto.
\end{itemize}

Os sapiens saíram do leste da África pela segunda vez com túmulos, pintando, com
capacidades sociais e mentais mais avançadas, porque tinham descoberto a
tecnologia espiritual.

\section{A espiritualidade como método de conquista da realidade}\label{met}

É abstração para treinamento neural. É repositório ancestral de meios para
ampliar a percepção e consciência dos atos e pensamentos, ou seja, para trazer à
consciência aspectos de antemão apenas acessados pelo insconsciente.

A natureza fragmentada do pensamento humano coloca a concepção da unidade como
algo transcendental, ligada às práticas meditativas e assumindo plena
visibilidade na figura do Deus monoteísta. Neste sentido, a concepção da figura
onipotente (onipresente e onisciente) é um curto-circuito de libido, permitindo
acessos profundos e eficientes ao inconsciente, muitas vezes com prazeres
intensos e transes.

\section{A espiritualidade como tecnologia ingênua}

Sistemas formais sem uma definição axiomática única são por vezes chamados
``ingênuos''. Neste mesmo sentido, os símbolos, rituais e sistematizações de
esferas espirituais caminham na direção de formalização e estabilização de
ideias sem que haja uma definição axiomática única ou mesmo nítida do sistema
considerado.

O desenvolvimento da espiritualidade como tecnologia ingênua é intenso desde sua
gênese com os sapiens. Suas bases empíricas e de transmissão para descendentes
são perfeitamente racionalizáveis e passíveis de aproveitamento consciente.

\section{Aproveitamento eficiente do arcabouço}

As próprias tradições possuem instruções para melhor aproveitamento do legado
espiritual. Dentre estas, constam:

\begin{itemize}
  \item Prática: a alocação de tempo para atividades espirituais desenvolve
  capacidades fluentes para seu aproveitamento.

  \item Aculturação: o conhecimento vinculado às tradições espirituais foi
  decantado por toda ancestralidade. É vantajoso o aprendizado constante deste
  legado.

  \item Conhecimento de diferentes tradições: facilita reconhecer instrumentais,
  pois existem constâncias. Ao mesmo tempo, evitam ressonâncias corporativistas
  e preconceituosas.
\end{itemize}

\end{document}
